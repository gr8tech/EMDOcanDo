\begin{table}[]
\centering
\caption{Basic Instructions for Control Loops}
\label{Basic_Instructions_for_Control_Loops}
\begin{tabular}{|p{4cm}|p{10cm}|}
\hline
\textbf{Basic Interpreter} & \textbf{Description}                                                     \\ \hline
\textbf{For} \textit{counter\%} = \textit{start\%} \textbf{To\%} \textit{finish} [\textbf{Step} \textit{increment\%}] \newline
[<statement>\newline
\textbf{Exit For}]\newline
\textbf{Next} [\textit{counter\%}, \textit{counter2\%}, ...]
& Initiates a \textbf{For}-\textbf{Next} loop with the \textit{counter\%} initially set to \textit{start\%} and incrementing in \textit{increment\%} steps (default is 1) until 'counter' equals \textit{finish\%}. The \textit{increment\%} must be an integer value, but may be negative.\newline
If multiple \textbf{For}-\textbf{Next} are cascaded, one \textbf{Next} statement can be used with the list of counter variables. It is recommended to state the counter variable for better readability of the code.\newline
\textbf{Exit For} provides a early exit for the loop. Execution will be continued after the \textbf{Next}.
\\ \hline
\textbf{Do}\newline
[<statements>\newline
\textbf{Exit Do}]\newline
\textbf{Loop}\newline
& This structure will loop forever; the \textbf{Exit Do} or only \textbf{xit} can be used to
terminate the loop or control must be explicitly transferred outside of the
loop by commands like \textbf{Goto} or \textbf{Return} (if in a subroutine).
\\ \hline
\textbf{Do While} \textit{expression}\newline
<statements>\newline
\textbf{Loop}\newline\newline
\textbf{Do}\newline
<statements>\newline
\textbf{Loop Until} \textit{expression}
& 
Loops while \textit{expression} is true (this is equivalent to the older \textbf{While}-
\textbf{Wend} loop, also implemented in MMBasic). \newline
If, at the start, the expression is false the statements in the loop will not be executed, even
once.\newline
Loops until the expression following \textbf{Until} is true. Because the test is
made at the end of the loop the statements inside the loop will be
executed at least once, even if the expression is false.
\\ \hline
\end{tabular}
\end{table}


\begin{table}[]
\centering
\caption{Basic Instructions for Control Flow Control If Switch}
\label{Basic_Instructions_for_Control_Flow_Control_If_Switch}
\begin{tabular}{|p{4cm}|p{10cm}|}
\hline
\textbf{Basic Interpreter} & \textbf{Description}                                                       \\ \hline
\textbf{If} \textit{expr} \textbf{Then} statement\newline
or\newline
\textbf{If} \textit{expr} \textbf{Then} statement\newline
\textbf{Else} statement
 & Evaluates the expression \textit{expr} and performs the \textbf{Then} statement if it is
true or skips to the next line if false. The optional \textbf{Else} statement is the
reverse of the \textbf{Then} test. This type of \textbf{If} statement is all on one line.
The \textbf{Then} statement construct can be also replaced with:
\textbf{Goto} \textit{linenumber} or \textbf{Goto} \textit{label}. A \textit{linenumber} is a increasing number that is add in front of each line of code. A \textit{label} is the keyword \textit{label} : (name followed by :) that is placed in front of a statement.
\\ \hline
\textbf{If} \textit{expr} \textbf{Then}\newline
<statements>\newline
[\textbf{Else}\newline
<statements>]\newline
[\textbf{Elseif} \textit{expr} \textbf{Then}\newline
<statements>]\newline
\textbf{Endif}
& The \textbf{Elseif} statement (if present) is executed if the previous condition
is false and it starts a new \textbf{If} chain with further \textbf{Else} and / or \textbf{Elseif} 
statements as required.\newline
One \textbf{Endif} is used to terminate the multiline \textbf{IF}. \newline
\textbf{Elseif} and \textbf{Else If} are equivalent.\newline
\textbf{Endif} and \textbf{End If} are equivalent.
\\ \hline
\textbf{Select Case} expression\newline
\textbf{Case} expression\newline
<statements>\newline
[\textbf{Case} expression\newline
<statements>\newline
\textbf{Case Else}\newline
<statements>]\newline
\textbf{End Select}
& The \textbf{Select Case} statement is similar to the \textbf{If} statement. It allows to compare an expression to one or more expressions. The optional \textbf{Case Else} matches if none of the previous \textbf{Case} statements matches.\newline 
The \textbf{Select Case} must always be terminated with a \textbf{End Select} statement.
\\ \hline
\end{tabular}
\end{table}

\begin{table}[]
\centering
\caption{Basic Instructions for Control Branches}
\label{Basic_Instructions_for_Control_Branches}
\begin{tabular}{|p{4cm}|p{10cm}|}
\hline
\textbf{Basic Interpreter} & \textbf{Description}                                                     \\ \hline
\textbf{Continue} & Resume running a program that has been stopped by an \textbf{End} statement,
an error, or web interface stop button. The program will restart with the next statement
following the previous stopping point.
\\ \hline
\textbf{Gosub} target & Initiates a subroutine call to the target, which can be a line number or a
label. The subroutine must end with \textbf{Return}.
\\ \hline
\textbf{Goto} target & Branches program execution to the target, which can be a line number or
a label.
\\ \hline
\textbf{On} expression \textbf{Gosub} target\newline 
\textbf{On} expression \textbf{Goto} target\newline 
\textbf{On CRON} clid\% handler [\textbf{Detach}]\newline 
\textbf{On TIMER} clid\% handler [\textbf{Detach}]\newline 
%\textbf{ON SEMP DEVCTL} fn\% handler [\textbf{DETACH}]\newline 
%\textbf{ON SEMP GET} fn\% handler [\textbf{DETACH}]\newline 
%\textbf{ON MSG} x\% handler [\textbf{DETACH}]\newline 
& If the expression evaluates to true, the \textbf{Gosub} / \textbf{Goto} is executed.\newline
Special functionality offers \textbf{On CRON} and \textbf{On TIMER}. This registers a handler that is executed if the cron or timer are scheduled. \newline
The optional \textbf{Detach} statement unregisters the handler again.
\\ \hline
\end{tabular}
\end{table}


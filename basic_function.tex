\begin{table}[]
\centering
\caption{Basic Instructions for Subroutines and Functions}
\label{Basic_Instructions_for_Subroutines_and_Functions}
\begin{tabular}{|p{4cm}|p{10cm}|}
\hline
\textbf{Basic Interpreter} & \textbf{Description}                                                         \\ \hline
\textbf{Function} \textit{function-name} ([\textit{arg1}, \textit{arg2}, …])\newline
[<statements>\newline
xxx = <return value>
\textbf{Exit Function}]\newline
\textbf{End Function}
& Defines a callable function. A function is a piece of code that takes some arguments and returns a result at the end.\newline
\textit{function-name} is the function name and it must meet the specifications for naming
a variable. \textit{arg1}, \textit{arg2}, etc are the arguments or parameters to the
function.\newline
To set the return value of the function you assign the value to the
function's name. For example:\newline
\textbf{Function} SQUARE(a)\newline
SQUARE = a * a\newline
\textbf{End Function}\newline
Every definition must have one \textbf{End Function} statement. When this
is reached the function will return its value to the expression from which
it was called. The command \textbf{Exit Function} can be used for an early
exit.\newline
You use the function by using its name and arguments in a program just
as you would a normal MMBasic function. For example:\newline
\textbf{Print} SQUARE(56.8)\newline
When the function is called each argument in the caller is matched to the
argument in the function definition. These arguments are available only
inside the function.\newline
Functions can be called with a variable number of arguments. Any
omitted arguments in the function's list will be set to zero or a null string.
Arguments in the caller's list that are a variable (ie, not an expression or
constant) will be passed by reference to the function. This means that
any changes to the corresponding argument in the function will also be
copied to the caller's variable.\newline
You must not jump into or out of a function using commands like
\textbf{Goto}, \textbf{Gosub}, etc. Doing so will have undefined side effects
including the possibility of ruining your day.
\\ \hline
\textbf{Sub} \textit{sub-name} ([\textit{arg1},\textit{arg2}, …])\newline
[<statements>\newline
\textbf{Exit Sub}]\newline
\textbf{End Sub}
& Defines a callable subroutine. This is the same as adding a new
command to MMBasic while it is running your program.\newline
\textit{sub-name} is the subroutine name and it must meet the specifications for
naming a variable. \textit{arg1}, \textit{arg2}, etc are the arguments or parameters to
the subroutine.\newline
Every definition must have one \textbf{End Sub} statement. When this is
reached the program will return to the next statement after the call to the
subroutine. The command \textbf{Exit Sub} can be used for an early exit.
You use the subroutine by using its name and arguments in a program
just as you would a normal command. \newline
For example: \newline
MySub a1, a2\newline
When the subroutine is called each argument in the caller is matched to
the argument in the subroutine definition. These arguments are available
only inside the subroutine. Subroutines can be called with a variable
number of arguments. Any omitted arguments in the subroutine's list
will be set to zero or a null string.\newline
Arguments in the caller's list that are a variable (ie, not an expression or
constant) will be passed by reference to the subroutine. This means that
any changes to the corresponding argument in the subroutine will also be
copied to the caller's variable and therefore may be accessed after the
subroutine has ended. 
\\ \hline
\end{tabular}
\end{table}
\begin{table}[]
\centering
\caption{Basic Instructions for IO Open and Close}
\label{Basic_Instructions_for_IO_Open_Close}
\begin{tabular}{|p{4cm}|p{10cm}|}
\hline
\textbf{Basic Interpreter} & \textbf{Description}                                                   \\ \hline
\textbf{OPEN} fname\$ \textbf{FOR} mode
\textbf{AS} [\#]fnbr
& Opens a file for reading or writing.
‘fname’ is the filename with maximal string length (case-sensitive). It can be prefixed with a directory path. For example: "DIR1/DIR2/FILE.EXT". SPIFFS does not support directories. Directories are only emulated. \newline
‘mode’ is \textbf{INPUT}, \textbf{OUTPUT}, \textbf{APPEND} or \textbf{RANDOM}.\newline\newline
\textbf{INPUT} will open the file for reading and throw an error if the file does
not exist. \newline
\textbf{OUTPUT} will open the file for writing and will automatically
overwrite any existing file with the same name.\newline
\textbf{APPEND} will also open the file for writing but it will not overwrite an
existing file; instead any writes will be appended to the end of the file. If
there is no existing file the APPEND mode will act the same as the
\textbf{OUTPUT} mode (i.e. the file is created then opened for writing). \newline
Note:\newline
\textbf{RANDOM} will open the file for both read and write and will allow
random access using the SEEK command. When opened the read/write
pointer is positioned at the end of the file.\newline
‘fnbr’ is the file number (1 to 10). The \# is optional. Up to 10 files can
be open simultaneously. The \textbf{INPUT}, \textbf{LINE INPUT}, \textbf{PRINT}, \textbf{WRITE}
and \textbf{CLOSE} commands as well as the \textbf{EOF}() and \textbf{INPUT\$}() functions all
use ‘fnbr’ to identify the file being operated on.\newline
See also OPTION ERROR and MM.ERRNO for error handling.
\\ \hline
\textbf{CLOSE} [\#]nbr [,[\#]nbr] … &
Close the file(s) previously opened with the file number ‘nbr’.
The \# is optional. Also see the \textbf{OPEN} command.
\\ \hline
\end{tabular}
\end{table}

\begin{table}[]
\centering
\caption{Basic Instructions for IO and File Operation}
\label{Basic_Instructions_for_IO_and_File_Operation}
\begin{tabular}{|p{4cm}|p{10cm}|}
\hline
\textbf{Basic Interpreter} & \textbf{Description}                                                 
\\ \hline
\textbf{INPUT} ["prompt string\$";]
list of variables
& Allows input from the keyboard to a list of variables. The input
command will prompt with a question mark (?).\newline
The input must contain commas to separate each data item if there is
more than one variable.\newline
For example, if the command is:\newline
\textbf{INPUT} a, b, c\newline
And the following is typed on the keyboard: 23, 87, 66\newline
Then a = 23 and b = 87 and c = 66\newline
If the "prompt string\$" is specified it will be printed before the question
mark. If the prompt string is terminated with a comma (,) rather than the
semicolon (;) the question mark will be suppressed.
\\ \hline
\textbf{INPUT}\$(nbr, [\#]fnbr) & Will return a string composed of ‘nbr’ characters read from a file
previously opened for INPUT with the file number ‘fnbr’. This function
will read all characters including carriage return and new line without
translation.\newline
When reading from a serial communications port this will return as
many characters as are waiting in the receive buffer up to ‘nbr’. If there
are no characters waiting it will immediately return with an empty string.\newline
The \# is optional. Also see the\textbf{OPEN} command.
\\ \hline
\textbf{INKEY}\$ & Checks the input of a character.
\\ \hline
\textbf{LINE INPUT} [prompt\$,] string-variable\$\newline
\textbf{LINE INPUT} \#nbr,
string-variable\$
 &  Reads entire line from the keyboard into ‘string-variable\$’. If specified
the ‘prompt\$’ will be printed first. Unlike \textbf{INPUT}, \textbf{LINE INPUT} will
read a whole line, not stopping for comma delimited data items.
A question mark is not printed unless it is part of ‘prompt\$’.
\\ \hline
\textbf{PRINT} expression [[,; ]expression] … etc\newline
\textbf{?} expression [[,; ]expression] … etc
& Outputs text to the browser. Multiple expressions can be used and must be
separated by either a: \newline
\textbf{?} Comma (,) which will output the tab character \newline
\textbf{?} Semicolon (;) which will not output anything (it is just used to separate expressions).\newline
\textbf{?} Nothing or a space which will act the same as a semicolon.\newline
A semicolon (;) at the end of the expression list will suppress the
automatic output of a carriage return/ newline at the end of a print
statement.\newline
When printed, a number is preceded with a space if positive or a minus
(-) if negative but is not followed by a space. Integers (whole numbers)
are printed without a decimal point while fractions are printed with the
decimal point and the significant decimal digits. Large numbers (greater
than six digits) are printed in scientific format.\newline
The function \textbf{FORMAT\$()} can be used to format numbers. The function
\textbf{TAB()} can be used to space to a certain column and the string functions
can be used to justify or otherwise format strings.
\\ \hline
\textbf{PRINT} \#nbr, expression
[[,; ]expression] … etc
& Same as above except that the output is directed to a file previously
opened for \textbf{OUTPUT} or \textbf{APPEND} as ‘nbr’. See the \textbf{OPEN} command.
\\ \hline
\end{tabular}
\end{table}

\begin{table}[]
\centering
\caption{Basic Instructions for SPIFFS File System}
\label{Basic_Instructions_for_SPIFFS_File_System}
\begin{tabular}{|p{4cm}|p{10cm}|}
\hline
\textbf{Basic Interpreter} & \textbf{Description}                                                   
\\ \hline
\textbf{Copy} src\$ To dest\$ & Copy the file named 'src\$' to another file named 'dest\$'.
\\ \hline
\textbf{Dir\$}( \textit{fspec} )\newline
\textit{file-name\$} = \textbf{Dir\$}( )\newline
& Will search the NOR flash for files and return the names of entries found.
\textit{fspec} is a file specification using wildcards the same as used by the
FILES command. E.g., "*.*" will return all entries, "*.TXT" will return
text files.\newline
The function without argument, will return the first entry found. To retrieve subsequent
entries use the function with no arguments. ie, \textbf{DIR\$}. The return of
an empty string indicates that there are no more entries to retrieve.\newline
This example will print all the files in a directory:\newline
\textit{f\$} = \textbf{Dir\$}("*.*")\newline
\textbf{Do While} \textit{f\$} <> ""\newline
\textbf{Print} f\$\newline
\textit{f\$} = \textbf{Dir\$}()\newline
\textbf{Loop}\newline
SPIFFS does not support directories. Leading path alike pattern will force search within a "directory".
\\ \hline
\textbf{Eof}( [\#]nbr ) & Will return true if the file previously opened for INPUT with the file
number ‘nbr’ is positioned at the end of the file.
If used on a file number opened as a serial port this function will return
true if there are no characters waiting in the receive buffer.
The \# is optional. Also see the OPEN, INPUT and LINE INPUT
commands and the \textbf{INPUT\$} function.
\\ \hline
\textbf{FILES} [fspec\$] 
&  Lists files in the current directory of the NOR flash.\newline
Optional use of ‘fspec\$’. Question marks
(?) will match any character and an asterisk (*) will match any number
of characters. If omitted, all files will be listed. For example:\newline
*.* Find all entries\newline
*.TXT Find all entries with an extension of TXT\newline
E*.* Find all entries starting with E\newline
X?X.* Find all three letter file names starting and ending with X
\\ \hline
\textbf{LOC}( [\#]fnbr ) & For a file opened as \textbf{RANDOM} this will return the current position of the read/write pointer in the file. Note that the first byte in a file is
numbered 1. The \# is optional.
\\ \hline
\textbf{LOF}( [\#]fnbr ) & For a file this will return the current length of the file in bytes.
The \# is optional.
\\ \hline
\textbf{KILL} file\$ & Deletes the file specified by ‘file\$’. If there is an extension it must be
specified.\newline
If 'file' is a string constant quote marks around it are optional at the
command prompt but must be specified if the command is used within a
program. Example: \textbf{KILL} “SAMPLE.DAT”
\\ \hline
\textbf{NAME} old\$ AS new\$ 
& Rename a file or a directory from ‘old\$’ to ‘new\$’
Unlike the other commands that work with file names the \textbf{NAME}
command cannot accept a full pathname (with directories).
\\ \hline
\\ \hline
\textbf{SEEK} [\#]fnbr, pos & Will position the read/write pointer in a file that has been opened for RANDOM access to the 'pos' byte.\newline
The first byte in a file is numbered one so \textbf{SEEK} \#5,1 will position the read/write pointer to the start of the file.
\\ \hline
\textbf{POS}() & Returns the current cursor position in the line in characters.
\end{tabular}
\end{table}


\begin{table}[]
\centering
\caption{Basic Instructions for S0 and Relay}
\label{Basic_Instructions_for_S0_relay}
\begin{tabular}{|p{4cm}|p{10cm}|}
\hline
\textbf{Basic Interpreter} & \textbf{Description}                                                   \\ \hline
\textit{pulse-count\%} = \textbf{S0In} ( \textit{port\%} [, \textit{info\$} )\newline
\textit{pulse-count\%} = \textbf{S0In} ( \textit{port\%} [, \textit{reset-counter\%} )
& The S0 and relay outputs are highly configurable to static and dynamic modes. Prior to using a \textit{port\%} it must be configured with \textbf{SYS.SET} for its operation.\newline
Examples:\newline
' TOn, TOff are the minimal on and off period of the pulse in ms. ThHi and ThLo the threshold values of the ADC for high and low signal detection. If the reset flag is set to 1, the current pulse counter is reset.\newline
\textbf{SYS.SET} "s0\_in0", "TOn=30 TOff=30 ThHi=700 ThLo=400 reset=1"\newline
' Read pulse-count from primary S0 input and reset the pulse-count.\newline
\textit{pulse-count\%} = \textbf{S0In}(0,1)\newline       
' Get static state of the first S0 input\newline
\textit{input-state} = \textbf{S0In}(0,'s')\newline
' Get pulse rise time of the first S0 input\newline
\textit{input-state} = \textbf{S0In}(0,'T')\newline
' Get pulse fall time of the first S0 input\newline
\textit{input-state} = \textbf{S0In}(0,'t')\newline
' Get average pulse on time of the first S0 input\newline
\textit{input-state} = \textbf{S0In}(0,'P')\newline
' Get average pulse off time of the first S0 input\newline
\textit{input-state} = \textbf{S0In}(0,'p')\newline
The average pulse times are interesting to calculate average value e.g. for power production or consumption from the counts.
\\ \hline
\textbf{S0Out} \textit{port\%}, \textit{pulse-count\%} [, \textit{blank-period\%}]
& The S0 and relay outputs are highly configurable to several static and dynamic modes. Prior to using a \textit{port\%} it must be configured with \textbf{SYS.SET} for its operation.\newline
Examples:\newline
' set first S0 output to 0 (phototransistor or relay driver conducting)\newline
\textbf{SYS.SET} "s0\_out0", "state=0"\newline
' set second S0 output to 1 (phototransistor or relay driver conducting)\newline
\textbf{SYS.SET} "s0\_out1", "state=1"\newline
' set second S0 output to pulsed mode with TOn and TOff period of 30m.\newline
\textbf{SYS.SET} "s0\_out1", "TOn=30000 TOff=30000 mode=1"\newline
' generate 100 pulses  (\textit{pulse-count\%}) on second S0 output\newline
\textbf{S0Out} 1,100\newline
' generate 2 pulses (\textit{pulse-count\%}) on second S0 output with 1 second \textit{blank-period}. This feature is usefull if the device receiving the pulse measures the time between the pulses to calculate a power value from it.\newline
\textbf{S0Out} 1,2, 1000
\\ \hline
\end{tabular}
\end{table}

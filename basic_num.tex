\begin{table}[]
\centering
\caption{Basic Instructions for Numerical Operation}
\label{Basic_Instructions_for_Numerical_Operation}
\begin{tabular}{|p{4cm}|p{10cm}|}
\hline
\textbf{Basic Interpreter} & \textbf{Description}                                                             \\ \hline
\textbf{ABS}( number ) 
& Returns the absolute value of the argument 'number' (ie, any negative sign is removed and the positive number is returned).
\\ \hline
\textbf{ACOS}( number ) 
& Returns the arccosinus of a number ( $0$ to $\pi$ ) in radians
\\ \hline
\textbf{COS}( number ) & Returns the cosine of the argument 'number' in radians.
\\ \hline
\textbf{ASIN}( number ) 
& Returns the arcsinus of a number in radians ( $-\frac{\pi}{2}$ to $\frac{\pi}{2}$ ) in radians
\\ \hline
\textbf{SIN}( number ) & Returns the sine of the argument 'number' in radians.
\\ \hline
\textbf{ATN}( number ) 
& Returns the arctangent of a number in radians ( $-\frac{\pi}{2}$ to $\frac{\pi}{2}$ ) in radians
\\ \hline
\textbf{TAN}( number ) & Returns the tangent of the argument 'number' in radians.
\\ \hline
\textbf{INT}( number ) & Truncate an expression to the next whole number less than or equal to
the argument. For example 9.89 will return 9 and -2.11 will return -3.
This behaviour is for Microsoft compatibility, the \textbf{FIX}() function
provides a true integer function.
See also \textbf{CINT}() .
\\ \hline
\textbf{CINT}( number ) & Round numbers with fractional portions up or down to the next whole number or integer.\newline
For example:\newline
45.47 will round to 45\newline
45.57 will round to 46\newline
-34.45 will round to -34\newline
-34.55 will round to -35\newline
See also \textbf{INT}() and \textbf{FIX}().
\\ \hline
\textbf{DEG}( radians ) & Converts 'radians' to degrees.
\\ \hline
\textbf{RAD}( degrees ) & Converts 'degrees' to radians.
\\ \hline
\textbf{EXP}( number ) & Returns the exponential value of 'number'.
\\ \hline
\textbf{FIX}( number ) & Truncate a number to a whole number by eliminating the decimal point
and all characters to the right of the decimal point.\newline
For example 9.89 will return 9 and -2.11 will return -2.\newline
The major difference between \textbf{FIX} and \textbf{INT} is that \textbf{FIX} provides a true
integer function (ie, does not return the next lower number for negative
numbers as \textbf{INT}() does). This behaviour is for Microsoft compatibility.
See also \textbf{CINT}() .
\\ \hline
\textbf{LOG}( number ) & Returns the natural logarithm of the argument 'number'. 
\\ \hline
\textbf{PI}() & Returns the value of pi.
\\ \hline
\textbf{RND}( number )\newline
\textbf{RND}
& Returns a pseudo-random number in the range of 0 to 0.999999. The
'number' value is ignored if supplied. This function is based on the EMDO crypto hardware module and does not need randomization.
\\ \hline
\textbf{ROUND}( number, places )
& Round the number to ‘number’ to ‘places’ number of fractional digits.
\\ \hline
\textbf{SGN}( number ) & Returns the sign of the argument 'number', +1 for positive numbers, 0 for
0, and -1 for negative numbers.
\\ \hline
\textbf{SQR}( number ) & Returns the square root of the argument 'number'.
\\ \hline
\end{tabular}
\end{table}

\begin{table}[]
\centering
\caption{Basic Instructions for Numerical Conversion}
\label{Basic_Instructions_for_Numerical_Conversion}
\begin{tabular}{|p{4cm}|p{10cm}|}
\hline
\textbf{Basic Interpreter} & \textbf{Description}                                                             \\ \hline
\textbf{BIN\$}( number ) 
& Returns a string giving the binary (base 2) value for the 'number'.
\\ \hline
\textbf{OCT\$}( number ) & Returns a string giving the octal (base 8) representation of 'number'.
\\ \hline
\textbf{HEX}\$( number ) & Returns a string giving the hexadecimal (base 16) value for the 'number'.
\\ \hline
\end{tabular}
\end{table}
